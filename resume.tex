%%%%%%%%%%%%%%%%%%%%%%%%%%%%%%%%%%%%%%%%%%%%%%%%%%%%%%%%%%%%%%%%%%%%%%%%
%%%%%%%%%%%%%%%%%%%%%% Simple LaTeX CV Template %%%%%%%%%%%%%%%%%%%%%%%%
%%%%%%%%%%%%%%%%%%%%%%%%%%%%%%%%%%%%%%%%%%%%%%%%%%%%%%%%%%%%%%%%%%%%%%%%

%%%%%%%%%%%%%%%%%%%%%%%%%%%%%%%%%%%%%%%%%%%%%%%%%%%%%%%%%%%%%%%%%%%%%%%%
%% NOTE: If you find that it says                                     %%
%%                                                                    %%
%%                           1 of ??                                  %%
%%                                                                    %%
%% at the bottom of your first page, this means that the AUX file     %%
%% was not available when you ran LaTeX on this source. Simply RERUN  %%
%% LaTeX to get the ``??'' replaced with the number of the last page  %%
%% of the document. The AUX file will be generated on the first run   %%
%% of LaTeX and used on the second run to fill in all of the          %%
%% references.                                                        %%
%%%%%%%%%%%%%%%%%%%%%%%%%%%%%%%%%%%%%%%%%%%%%%%%%%%%%%%%%%%%%%%%%%%%%%%%

%%%%%%%%%%%%%%%%%%%%%%%%%%%% Document Setup %%%%%%%%%%%%%%%%%%%%%%%%%%%%

% Don't like 10pt? Try 11pt or 12pt
\documentclass[10pt]{article}

% This is a helpful package that puts math inside length specifications
\usepackage{calc}


% Simpler bibsection for CV sections
% (thanks to natbib for inspiration)
\makeatletter
\newlength{\bibhang}
\setlength{\bibhang}{1em}
\newlength{\bibsep}
 {\@listi \global\bibsep\itemsep \global\advance\bibsep by\parsep}
\newenvironment{bibsection}%
        {\vspace{-\baselineskip}\begin{list}{}{%
       \setlength{\leftmargin}{\bibhang}%
       \setlength{\itemindent}{-\leftmargin}%
       \setlength{\itemsep}{\bibsep}%
       \setlength{\parsep}{\z@}%
        \setlength{\partopsep}{0pt}%
        \setlength{\topsep}{0pt}}}
        {\end{list}\vspace{-.6\baselineskip}}
\makeatother

% Layout: Puts the section titles on left side of page
\reversemarginpar

%
%         PAPER SIZE, PAGE NUMBER, AND DOCUMENT LAYOUT NOTES:
%
% The next \usepackage line changes the layout for CV style section
% headings as marginal notes. It also sets up the paper size as either
% letter or A4. By default, letter was used. If A4 paper is desired,
% comment out the letterpaper lines and uncomment the a4paper lines.
%
% As you can see, the margin widths and section title widths can be
% easily adjusted.
%
% ALSO: Notice that the includefoot option can be commented OUT in order
% to put the PAGE NUMBER *IN* the bottom margin. This will make the
% effective text area larger.
%
% IF YOU WISH TO REMOVE THE ``of LASTPAGE'' next to each page number,
% see the note about the +LP and -LP lines below. Comment out the +LP
% and uncomment the -LP.
%
% IF YOU WISH TO REMOVE PAGE NUMBERS, be sure that the includefoot line
% is uncommented and ALSO uncomment the \pagestyle{empty} a few lines
% below.
%

%% Use these lines for letter-sized paper
\usepackage[paper=letterpaper,
            %includefoot, % Uncomment to put page number above margin
            marginparwidth=1.2in,     % Length of section titles
            marginparsep=.05in,       % Space between titles and text
            margin=1in,               % 1 inch margins
            includemp]{geometry}

%% Use these lines for A4-sized paper
%\usepackage[paper=a4paper,
%            %includefoot, % Uncomment to put page number above margin
%            marginparwidth=30.5mm,    % Length of section titles
%            marginparsep=1.5mm,       % Space between titles and text
%            margin=25mm,              % 25mm margins
%            includemp]{geometry}

%% More layout: Get rid of indenting throughout entire document
\setlength{\parindent}{0in}

%% This gives us fun enumeration environments. compactitem will be nice.
\usepackage{paralist}

%% Reference the last page in the page number
%
% NOTE: comment the +LP line and uncomment the -LP line to have page
%       numbers without the ``of ##'' last page reference)
%
% NOTE: uncomment the \pagestyle{empty} line to get rid of all page
%       numbers (make sure includefoot is commented out above)
%
\usepackage{fancyhdr,lastpage}
\pagestyle{fancy}
%\pagestyle{empty}      % Uncomment this to get rid of page numbers
\fancyhf{}\renewcommand{\headrulewidth}{0pt}
\fancyfootoffset{\marginparsep+\marginparwidth}
\newlength{\footpageshift}
\setlength{\footpageshift}
          {0.5\textwidth+0.5\marginparsep+0.5\marginparwidth-2in}
\lfoot{\hspace{\footpageshift}%
       \parbox{4in}{\, \hfill %
                    \arabic{page} of \protect\pageref*{LastPage} % +LP
%                    \arabic{page}                               % -LP
                    \hfill \,}}

% Finally, give us PDF bookmarks
\usepackage{color,hyperref}
\definecolor{darkblue}{rgb}{0.0,0.0,0.3}
\hypersetup{colorlinks,breaklinks,
            linkcolor=darkblue,urlcolor=darkblue,
            anchorcolor=darkblue,citecolor=darkblue}

%%%%%%%%%%%%%%%%%%%%%%%% End Document Setup %%%%%%%%%%%%%%%%%%%%%%%%%%%%


%%%%%%%%%%%%%%%%%%%%%%%%%%% Helper Commands %%%%%%%%%%%%%%%%%%%%%%%%%%%%

% The title (name) with a horizontal rule under it
%
% Usage: \makeheading{name}
%
% Place at top of document. It should be the first thing.
\newcommand{\makeheading}[1]%
        {\hspace*{-\marginparsep minus \marginparwidth}%
         \begin{minipage}[t]{\textwidth+\marginparwidth+\marginparsep}%
                {\large \bfseries #1}\\[-0.15\baselineskip]%
                 \rule{\columnwidth}{1pt}%
         \end{minipage}}

% The section headings
%
% Usage: \section{section name}
%
% Follow this section IMMEDIATELY with the first line of the section
% text. Do not put whitespace in between. That is, do this:
%
%       \section{My Information}
%       Here is my information.
%
% and NOT this:
%
%       \section{My Information}
%
%       Here is my information.
%
% Otherwise the top of the section header will not line up with the top
% of the section. Of course, using a single comment character (%) on
% empty lines allows for the function of the first example with the
% readability of the second example.
\renewcommand{\section}[2]%
        {\pagebreak[3]\vspace{1.3\baselineskip}%
         \phantomsection\addcontentsline{toc}{section}{#1}%
         \hspace{0in}%
         \marginpar{
         \raggedright \scshape #1}#2}

% An itemize-style list with lots of space between items
\newenvironment{outerlist}[1][\enskip\textbullet]%
        {\begin{itemize}[#1]}{\end{itemize}%
         \vspace{-.6\baselineskip}}

% An environment IDENTICAL to outerlist that has better pre-list spacing
% when used as the first thing in a \section
\newenvironment{lonelist}[1][\enskip\textbullet]%
        {\vspace{-\baselineskip}\begin{list}{#1}{%
        \setlength{\partopsep}{0pt}%
        \setlength{\topsep}{0pt}}}
        {\end{list}\vspace{-.6\baselineskip}}

% An itemize-style list with little space between items
\newenvironment{innerlist}[1][\enskip\textbullet]%
        {\begin{compactitem}[#1]}{\end{compactitem}}

% An environment IDENTICAL to innerlist that has better pre-list spacing
% when used as the first thing in a \section
\newenvironment{loneinnerlist}[1][\enskip\textbullet]%
        {\vspace{-\baselineskip}\begin{compactitem}[#1]}
        {\end{compactitem}\vspace{-.6\baselineskip}}

% To add some paragraph space between lines.
% This also tells LaTeX to preferably break a page on one of these gaps
% if there is a needed pagebreak nearby.
\newcommand{\blankline}{\quad\pagebreak[2]}

% Uses hyperref to link DOI
\newcommand\doilink[1]{\href{http://dx.doi.org/#1}{#1}}
\newcommand\doi[1]{doi:\doilink{#1}}

% For \url{SOME_URL}, links SOME_URL to the url SOME_URL
\providecommand*\url[1]{\href{#1}{#1}}
% Same as above, but pretty-prints SOME_URL in teletype fixed-width font
\renewcommand*\url[1]{\href{#1}{\texttt{#1}}}

% For \email{ADDRESS}, links ADDRESS to the url mailto:ADDRESS
\providecommand*\email[1]{\href{mailto:#1}{#1}}
% Same as above, but pretty-prints ADDRESS in teletype fixed-width font
%\renewcommand*\email[1]{\href{mailto:#1}{\texttt{#1}}}

%%%%%%%%%%%%%%%%%%%%%%%% End Helper Commands %%%%%%%%%%%%%%%%%%%%%%%%%%%

%%%%%%%%%%%%%%%%%%%%%%%%% Begin CV Document %%%%%%%%%%%%%%%%%%%%%%%%%%%%

\begin{document}
\makeheading{Gabriel Isman}

\section{Contact Information}
%
% NOTE: Mind where the & separators and \\ breaks are in the following
%       table.
%
% ALSO: \rcollength is the width of the right column of the table
%       (adjust it to your liking; default is 1.85in).
%
\newlength{\rcollength}\setlength{\rcollength}{1.85in}%
%
\begin{tabular}[t]{@{}p{\textwidth-\rcollength}p{\rcollength}}
Gabe Isman
                           & \textit{Phone:} +1-413-559-0108 \\
80 Lexington Ave. \#203           & \email{gabe.isman@gmail.com}\\
Brooklyn, New York 11238 & \href{https://github.com/GabeIsman}{github.com/GabeIsman} \\
\end{tabular}

\section{Summary}
%
Full-stack web developer looking for work on an interesting project, with a great
team, and plentiful opportunities for learning and growth.

\section{Education}
%
\href{http://www.oberlin.edu/}{\textbf{Oberlin College}},
Oberlin, Ohio USA
        \begin{innerlist}
        \item BA in \href{http://www.cs.oberlin.edu}{Computer Science}, minors in Math and Philosophy
        \item GPA: 3.90
        \end{innerlist}

        \section{Technical\\Skills}
%
Languages: PHP, JavaScript, SQL, Ruby, HTML, CSS

Frameworks: Yii, Backbone, Rails, JQuery

Applications: Git, MySQL, nginx, memcached, Redis, Sass, node

Services: AWS (OpsWorks, EC2, S3, RDS, ElasticBeanstalk)

\section{Work Experience}
%
\textbf{\href{http://www.skillshare.com/}{Skillshare}},
\begin{outerlist}

\item[] \textit{Software Engineer} \hfill \textbf{June 2012 to July 2013}
    \begin{innerlist}
        \item Was hired to build out Skillshare's PHP/Yii/MySQL back-end.
        Transitioned to a full-stack dev as the app evolved to be client-driven.
        \item Cut application server response time by 40\% through query optimization,
        caching strategies, and application architecture modifications.
        \item Migrated back-end infrastructure from AWS ElasticBeanstalk to OpsWorks.
        \item Migrated several key components of the product to be  rendered
        client-side.
        \item Worked collaboratively on a small product team, both leading features
        and supporting teammates with ideas and input, iterating very quickly as we
        developed and refined a new product.
        \item More details available on my github \href{http://github.com/gabeisman/resume}{resume}
    \end{innerlist}
\end{outerlist}

\blankline

\textbf{\href{http://www.liveramp.com/}{Rapleaf (now LiveRamp)}},
\begin{outerlist}

\item[] \textit{Software Engineering Intern} \hfill \textbf{Summer 2011}
\begin{innerlist}
	\item Interned on the web team at Rapleaf, a big-data, ad-tech startup. I
    developed and deployed a variety of consumer, customer, and internal tools.
    I worked across their Ruby/Rails/SQL/JQuery/Sass stack.
\end{innerlist}
\end{outerlist}

\blankline

\textbf{\href{http://www.cs.oberlin.edu/}{Oberlin Computer Science Department}},
\begin{outerlist}

\item[] \textit{Research Assistant} \hfill \textbf{Summer 2010}
\begin{innerlist}
	\item Did research in algorithmic game theory with Professor Tom Wexler, and others of Oberlin and Carleton Colleges.
\end{innerlist}
\item[] \textit{TA, Grader, and Tutor}%
        \hfill \textbf{September 2009 to May 2012}
\end{outerlist}

\blankline

\blankline

\href{http://www.eos.unh.edu/}{\textbf{Institute for the Study of Earth, Oceans, and Space}},
Durham, NH
\begin{outerlist}

\item[] \textit{Research Assistant to Charles W. Smith}%
        \hfill \textbf{June 2007 to January 2008}
\begin{innerlist}
    \item Assisted Professor Smith with various menial tasks associated with
    processing data from the ACE satellite during solar events.
\end{innerlist}

\end{outerlist}

\blankline


\section{Conference\\Publications} \begin{bibsection}
  \item Smith, Charles W.; Lee, Martin A.; Haggerty, Dennis K. et al. Preliminary Results from SEP and ESP Studies in \emph{PARTICLE ACCELERATION AND TRANSPORT IN THE HELIOSPHERE AND BEYOND: 7th Annual International Astrophysics Conference. AIP Conference Proceedings}, Volume 1039, pp. 168-173. 2008.
\end{bibsection}


\section{Honors}
%
 National Science Foundation S-STEM Scholar in Computation and Modeling at Oberlin College, 2010 -- 2012 \\


John F. Oberlin Scholar, 2008 -- 2012



\section{References Available on Request}

%
% \href
% {http://www.skillshare.com/about/team}
% {\textbf{Eric Ma}}
% (e-mail:~\href{mailto:eric@skillshare.com}{eric@skillshare.com})
% %
% \begin{innerlist}
%     \item Lead Developer,
%         \href{http://www.skillshare.com/}{Skillshare}

%     \item \emph{Eric is my direct manager and leads the engineering team.}
% \end{innerlist}

% \blankline

% %
% \href
% {http://www.skillshare.com/about/team}
% {\textbf{Andrew Mercando}}
% (e-mail:~\href{mailto:andrew@skillshare.com}{andrew@skillshare.com})
% %
% \begin{innerlist}
%     \item Lead Designer,
%         \href{http://www.skillshare.com/}{Skillshare}

%     \item \emph{Andrew and I have worked closely on many projects.}
% \end{innerlist}

% \blankline

% \href{}
% {\textbf{Anneka Gupta}}
% (e-mail:~\href{mailto:anneka@rapleaf.com}{anneka@rapleaf.com})
% \begin{innerlist}
%     \item Product Manager at LiveRamp
%     \item\emph{Anneka was my manager and mentor for the summer I worked at Rapleaf.}
% \end{innerlist}

\blankline

\end{document}

%%%%%%%%%%%%%%%%%%%%%%%%%% End CV Document %%%%%%%%%%%%%%%%%%%%%%%%%%%%%
